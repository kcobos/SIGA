\begin{center}
	{\large\bfseries Localizador de Aparcamiento para Personas con Discapacidad}\\
\end{center}
\begin{center}
	Carlos Cobos Suárez\\
\end{center}

\vspace{0.7cm}
\noindent{\textbf{Palabras clave}: accesibilidad, aparcamiento, geoposicionamiento, aplicación móvil, Android, wemos, Arduino, sensores, Python, CherryPy, API REST, aplicación web, AngularJS}\\

\vspace{0.7cm}
\noindent{\textbf{Resumen}}\\

El presente proyecto va a estudiar un problema real de la sociedad, los aparcamientos para personas con movilidad reducida, para proponer una solución específica. Esta solución abarcará todo lo posible para mejorar el funcionamiento de estas plazas reservadas, al diseñar un sistema de gestión integral y automático.
\\\\
Tras analizar el problema en su totalidad, se diseña e implementa un conjunto de funcionalidades en el ámbito del TFG. Con esto se consigue mostrar la viabilidad de la propuesta y aportar un núcleo que sirva de base para una futura implantación real del proyecto.
\\\\
Este núcleo tiene componentes hardware que habrá que diseñar, dos aplicaciones de usuario y una aplicación de servidor que orquesta todo el sistema. Para realizar las distintas aplicaciones, así como el componente hardware, se elegirán tecnologías abiertas y gratuitas pensando en una futura implantación del sistema.

