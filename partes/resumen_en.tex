\begin{center}
	{\large\bfseries Localizador de Aparcamiento para Personas con Discapacidad}\\
\end{center}
\begin{center}
	Carlos Cobos Suárez\\
\end{center}

\vspace{0.7cm}
\noindent{\textbf{Keywords}: accesibility, parking, geolocation, mobile app, Android, wemos, Arduino, sensors, Python, CherryPy, API REST, web app, AngularJS}\\

\vspace{0.7cm}
\noindent{\textbf{Abstract}}\\

This project will study a real problem in society, parking for people with reduced mobility, to propose a specific solution. This solution will cover everything possible to improve the operation of these reserved spaces, by designing an integral and automatic management system.
\\\\
After analyzing the problem in its entirety, a set of functionalities have been designed and implemented in the scope of a final project. With this, it has been possible to show the viability of the proposal and provide a core that could be used as the basis for a future real implementation of the project.
\\\\
This core has hardware components that will have to be designed, two user applications and a server application that guides the entire system. In order to carry out the different applications, as well as the hardware component, open and free technologies will be chosen with a view to future implementation of the system.
