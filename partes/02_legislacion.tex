\chapter{Aparcamientos para PMR en España}
Antes de comenzar a pensar en una solución que controle el uso de las plazas reservadas para personas con movilidad reducida (PMR), es fundamental analizar el problema real, así como su marco legal para que sirva de base en la creación de una idea que solucione el problema.
\\\\
Como se ha dicho en la introducción, existen plazas de aparcamiento reservadas para PMR en los municipios. Para hacer uso de las mismas, es necesario mostrar en la parte delantera del vehículo que haya estacionado una tarjeta acreditativa. Esta tarjeta, al igual que las plazas, está regulada.
\\\\
El organismo precursor que promovió el uso de la tarjeta acreditativa de estacionamiento para PMR fue la Unión Europea a través de la recomendación que hizo en 1998 \cite{ue-98}. A partir de entonces, y tal y como expone dicha recomendación, los diferentes países de la UE deberían reconocer este tipo de tarjetas.
\\\\
En España, el último decreto que regula el uso y emisión de la tarjeta de estacionamiento para personas con discapacidad es el Real Decreto 1056/2014, de 12 de diciembre \cite{rd1056-2014}. En este decreto se expone qué es la tarjeta de estacionamiento, quién puede poseer una, cuál es el ámbito de la misma, así como el porcentaje de plazas reservadas.
\\\\
No obstante, desde el año 1999, por recomendación de la UE, las comunidades autónomas empezaron a regular las plazas reservadas para PMR como se puede ver en el punto 2 del artículo 51 de la Ley 1/1999, de 31 de marzo, de Atención a las Personas con Discapacidad en Andalucía. A su vez, el Gobierno Español en el artículo 30 del Real Decreto Legislativo 1/2013, de 29 de noviembre, por el que se aprueba el Texto Refundido de la Ley General de derechos de las personas con discapacidad y de su inclusión social \cite{rdl1-2013}, se dice que los ayuntamientos de cada municipio tendrán que reservar plazas de aparcamiento para PMR.
\\\\
Recientemente, la Ley 1/1999, de 31 de marzo, de Atención a las Personas con Discapacidad en Andalucía, se ha visto derogada por la Ley 4/2017, de 25 de septiembre, de los Derechos y la Atención a las Personas con Discapacidad en Andalucía \cite{l4-2017}, en la cual se corrobora, en su artículo 55, que se tiene que reservar un porcentaje de plazas de aparcamiento para personas que posean una tarjeta de aparcamiento y dichas plazas las tiene que dotar el ayuntamiento en cada municipio. También, en su artículo 56, se indica que la Consejería competente en materia de servicios sociales regulará el procedimiento de reconocimiento y concesión de la tarjeta de aparcamiento a las personas con discapacidad. Esto quiere decir que son las comunidades autónomas las encargadas de expedir las tarjetas de aparcamiento a PMR.
\\\\
Por otro lado, ese mismo artículo expone que la Policía Local de cada municipio será la encargada de controlar el uso adecuado de la tarjeta de aparcamiento. A su vez, el Real Decreto Legislativo 6/2015, de 30 de octubre, por el que se aprueba el texto refundido de la Ley sobre Tráfico, Circulación de Vehículos a Motor y Seguridad Vial \cite{rdl6-2015} expone como infracción grave, en su artículo 76 punto d, parar o estacionar en plazas para PMR, así como proceder a su retirada de conformidad al punto e del artículo 105, cuando el vehículo no disponga de distintivo (tarjeta de aparcamiento).
\\\\
En definitiva, existe un marco legal bastante sólido proveniente de la Unión Europea que se ha adaptado en España a través de las comunidades autónomas. En otros países de la UE la adaptación tiene ser parecida a la nuestra. Esto supone que el presente proyecto se pueda extrapolar a más países además de España. También, como se puede comprobar, existe un marco legal bastante importante que da sustento a poder hacer este proyecto.
\\\\
Este marco legal, junto con la exposición de problemáticas que hay entorno a las plazas reservadas para PMR, sirve de aliciente para empezar a trabajar en un proyecto que solucione los distintos problemas que hay en dichas plazas.