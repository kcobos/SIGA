\chapter{Introducción}
Debido a la tendencia de tener un vehículo por habitante y a centralizar todo en las ciudades, resulta muy difícil encontrar una plaza de aparcamiento para estacionar un vehículo particular. Además, desplazarse por la ciudad supone un gasto de tiempo que va en aumento causado por el tráfico. Esto no sólo supone un problema para la persona que intenta desplazarse con su vehículo sino que, a su vez, afecta a todos los ciudadanos a través de la contaminación de los vehículos.
\\\\
No obstante, actualmente se está produciendo un cambio en la sociedad para pasar de desplazase en un trasporte privado a uno público o sostenible por la ciudad. Se está apostando por el transporte público para así evitar atascos o no tener que buscar un aparcamiento, aunque no se pueda ir al lugar exacto de destino (siempre hay que andar). Otro porcentaje de ciudadanos optan por un transporte sostenible, como el de una bicicleta, para desplazarse por la ciudad. Esta última elección podría ser más cómoda debido a que se evitan las esperas para hacer uso del autobús, metro…, y, a su vez, al ser un medio privado, la persona puede ir exactamente a su destino.
\\\\
Desafortunadamente, ambas opciones alternativas de transporte no pueden ser siempre viables. En la sociedad actual existe un colectivo de personas que, por dificultades físicas, con frecuencia no pueden renunciar a un medio de transporte privado para desplazarse por la ciudad. Estas personas serían aquellas personas mayores que presentan problemas para desplazarse o personas con algún tipo de discapacidad. Es por esto que este colectivo necesita disponer de un espacio de aparcamiento cercano a lugares de interés como centros de ocio, hospitales, universidades, colegios…, o a su propia vivienda, ya que supone un 8,5\% \cite{ine_discapacidad_2008} de la población española. 
\\\\
Dicho colectivo es cada vez más numeroso debido a que la esperanza de vida va aumentando con el paso del tiempo fruto de los avances en el ámbito sanitario. Esto hace que cada vez haya un número mayor de personas mayores, personas con algún tipo de enfermedad rara, o discapacitados en general que puedan encontrarse en esta situación. Es por esto que, aunque haya plazas de aparcamiento reservadas para este colectivo, suelen no ser suficientes.
\\\\
En cualquier caso, a día de hoy, este tipo de plazas es la solución que la Administración propone para facilitar, a personas con movilidad reducida (PMR), el aparcamiento en la ciudad. Dichas plazas presentan una señal indicativa prohibiendo el estacionamiento a todo aquel vehículo que no presente una tarjeta acreditativa para hacer uso de las mismas. Ello supone que si hubiese un vehículo aparcado en estas plazas sin tarjeta, la autoridad competente sancionaría al propietario y, si corresponde, retiraría el vehículo.  Este control en muchas ocasiones es deficiente.  Por tanto, la cantidad limitada de plazas disponible se ve reducida por un mal uso de las mismas.
\\\\
Este hecho hace que aquella persona con movilidad reducida que necesite aparcar en una plaza ocupada por un vehículo sin acreditación, se ve obligada a aparcar más lejos de su destino o tener que llamar a la autoridad para que retiren el vehículo de la plaza reservada. Ambas acciones suponen una pérdida de tiempo para la persona que necesita hacer uso, de forma correcta, de la plaza.
\\\\
La solución al problema, pasa necesariamente por un aumento del número de plazas y un uso más responsable de las mismas inducido o acompañado por un control adecuado. Mientras llega, sería de mucho interés un sistema que controle la ocupación de este tipo de plazas, avisando, si es necesario, a la autoridad para que procedan a la sanción y/o retirada de un vehículo. De esta manera se podría asegurar que cualquier persona que pueda hacer uso de estas plazas, solamente tenga que buscar un lugar para aparcar más lejano cuando estén lícitamente ocupadas, además de liberar a la autoridad de una tarea repetitiva.  Al tiempo que se facilita el acceso a estas plazas al colectivo destinado, se disminuye el gasto público que supone la necesidad de personal para el control de las plazas o el impacto que puede tener dicho control en la respuesta de la autoridad frente a otro tipo de incidente. 
\\\\
Como precedente, en España en 2014, hubo una iniciativa social - Disabled Park \cite{disabledPark} - con plataforma web y aplicación móvil con el fin de geolocalizar los aparcamientos y avisar a la autoridad sobre el uso indebido de una plaza. Esta iniciativa no ha tenido mucha repercusión debido a que necesita “padrinos”, es decir, terceras personas que controlan las plazas. El punto positivo es que, al tener las zonas de aparcamiento localizadas, en la aplicación móvil o en la plataforma web, el usuario puede visualizar las plazas cercanas en el mapa, así como saber dónde se encuentran. 
\\\\
Por otro lado, en Italia ese mismo año, se empezó a desarrollar un dispositivo electrónico llamado Tommy \cite{tommy}, para detectar si un vehículo sin autorización aparca en una plaza reservada al lado de la vivienda de un usuario. Este dispositivo electrónico ha tenido muchos obstáculos legales. En primera instancia, se querían poner bolardos o barreras controladas por el dispositivo para así imposibilitar el aparcamiento a vehículos sin acreditación, idea que fue rechazada legalmente. Posteriormente, tras descubrir los impedimentos legales de la primera versión, se pensó en una nueva en la que se avisa a través de una alarma cuando un vehículo sin autorización aparca en una zona reservada. Esta última versión avisa al usuario de la plaza a través de un SMS si la plaza queda ocupada, dejando al usuario la responsabilidad de llamar a la autoridad. 
\\\\
Como se acaba de ver, existen iniciativas para controlar el estacionamiento en este tipo de plazas reservadas pero, por diferentes motivos, no se han podido implantar o no cumplen completamente con su función. Este proyecto se centra en un sistema automático para gestionar este tipo de plazas, incluyendo entre sus objetivos el control de su uso correcto. Se estudiará de qué manera el sistema puede detectar el estado de una plaza de manera legal. Para ello se necesita estudiar el marco legal que afecta a este tipo de plazas. Además, aprovechando el tema jurídico, como el sistema tendrá que trabajar con datos calificados de carácter personal, se tendrá en cuenta toda la legislación vigente al analizar y desarrollar el sistema. Para ello se disgregarán los datos de los usuarios del sistema trabajando solamente con un identificador propio.
\\\\
Utilizando el sistema de control de las plazas y debido a que éstas tienen que estar geolocalizadas, es decir, posicionadas en un mapa, los usuarios de estas plazas podrían ver el estado en tiempo real de las mismas. Así, buscar aparcamiento cerca de una zona determinada sería más sencillo e incluso, esta acción se podría automatizar haciendo que el sistema busque la plaza más cercana dado un destino. Este será el segundo gran objetivo de nuestro sistema.
\\\\
Adicionalmente, el sistema que en este proyecto se va a elaborar se quiere que sea un sistema con una futura implantación en ciudades o municipios, ya sea en España o en el resto de países. Por tanto, no sólo deberá ceñirse al aspecto técnico sino también a aspectos legales, presupuestarios y accesibles. A \textit{aspectos legales} como los anteriormente citados; \textit{presupuestarios} debido a que si se quiere implantar, el sistema debería tener un presupuesto máximo razonable; y \textit{accesibles}, ya que todos los usuarios de estas plazas usarán el sistema para localizar un aparcamiento. Este uso puede ser en el vehículo, con lo que la aplicación tiene que ser lo más fácil e intuitiva posible.
\\\\
\newpage
Para poder llevar a cabo la creación de este sistema de gestión automática (en adelante, SGA) de aparcamientos para personas con movilidad reducida, así como una aplicación que permita a los usuarios localizar los aparcamientos, se deberá:
\begin{itemize}
	\item Analizar en detalle el problema real. Como se ha mencionado antes, hay que definir un marco legal donde el proyecto tenga soporte para una futura implantación así como analizar qué serie de procedimientos realizar al conocer un aparcamiento indebido.
	\item Proponer una solución de SGA  que satisfaga todas las especificaciones deseables. Dicha propuesta deberá incluir un sistema para saber el estado de ocupación de las plazas de aparcamiento, un sistema de gestión y recepción de alertas, así como una aplicación móvil para el usuario.
	\item Adaptar la solución antes descrita para que sea viable dentro de un TFG debido a la complejidad y tamaño que puede tener la solución propuesta. En esta adaptación se deberá simplificar o escoger partes de ésta para que se pueda implementar en tiempo y forma dentro de un TFG, haciendo, en cualquier caso, un sistema que funcione íntegramente y muestre la viabilidad de la propuesta.
	\item Analizar la adaptación del problema desde un punto de vista técnico. Una vez que se elijan qué características implementar en este TFG, hay que analizar dichas características técnicamente argumentado de forma detallada cómo se podrían desarrollar y qué tecnologías usar.
	\item Por último, una vez redactados todos los requisitos técnicos, implementar y desplegar dicha adaptación. Esta será un prototipo del sistema ideal para una futura muestra del sistema a La Administración Pública.
\end{itemize}
