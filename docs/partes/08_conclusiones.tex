\chapter{Conclusiones y Trabajo Futuro}
Desde el punto de vista de la solución alcanzada, como se ha podido ver en este trabajo, se han desarrollado tres prototipos software (API REST, aplicación de administración y aplicación de usuario) y uno hardware (agente de captación de datos). El sistema que se ha desarrollado es una prueba de concepto y, a su vez, un producto mínimo viable (PMV) que aúne estos cuatro prototipos para dar solución a un problema real de la sociedad.
\\\\
Desde el punto de vista académico, por un lado se ha puesto de manifiesto que se ha analizado un problema real junto con sus restricciones y se ha combinado el conocimiento con la técnica adecuada para proponer un sistema completo de tal calibre. Como dicho análisis queda grande en el contexto de un TFG, se ha procedido a reducir dicho sistema completo en otro que, a su vez, es completamente funcional.
\\\\
Por otro lado, la implementación y la creación del prototipo hardware han sido posibles tras un minucioso estudio de distintas tecnologías dentro del ámbito TICs. A su vez, se ha querido encontrar aquellas tecnologías que resuelven el problema de la mejor manera posible sin necesitar demasiados recursos. Esto se traduce en que se ha puesto un elevado interés para intentar que el sistema desarrollado sea viable en un futuro.
\\\\
También, desde el punto de vista personal, la creación de este proyecto ha sido una gran satisfacción al ver que he tenido los conocimientos, técnicas y autodeterminación necesarias para poder abordar un problema real que, al igual que a mí, afecta a un colectivo de personas. Dicho esto último, tras analizar soluciones similares, haber creado un PMV puede abrirme un futuro laboral inesperado.
\\\\
Sabiendo que se ha tenido que hacer una adaptación, que realmente es el núcleo principal, de un sistema ideal de gestión de aparcamientos para PMR, ha quedado un gran desarrollo para el futuro. Este trabajo no es sólo de desarrollo de algunas funcionalidades que, por el marco de este trabajo, no se han podido implementar sino, a su vez, se ha quedado un trabajo de investigación para predecir el estado de las ubicaciones para una mejor experiencia del usuario.
\\\\
Además, debido a que se ha estudiado el marco legal del problema a nivel Europeo, esta solución puede expandirse a otros países dando pie a un desarrollo todavía mayor.
