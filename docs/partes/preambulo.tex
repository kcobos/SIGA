\documentclass[paper=a4, fontsize=11pt]{book}
\makeatletter
\renewcommand*\cleardoublepage{\clearpage\if@twoside
	\ifodd\c@page \hbox{}\newpage\if@twocolumn\hbox{}%
	\newpage\fi\fi\fi}
\makeatother
% ---- Entrada y salida de texto -----

\usepackage[T1]{fontenc}
\usepackage[utf8]{inputenc}

% ---- Idioma --------

\usepackage[spanish, es-tabla]{babel}

% ---- Otros paquetes ----
\usepackage[hidelinks]{hyperref}
\usepackage{graphicx}
\usepackage{rotating}

\usepackage{longtable, tabularx}
\usepackage{booktabs}
\usepackage{ltablex}
\usepackage{array}
\newcommand{\anchoColumna}{7cm}
\newcommand{\anchoColumnaInterior}{6.1cm}
\newcommand{\anchoColumnaMasInterior}{5cm}


%\usepackage{algorithm2e}
%\renewcommand{\algorithmcfname}{Pseudo-código}
%\renewcommand*{\listalgorithmcfname}{Índice de pseudo-códigos}
%\usepackage{listings}
%\renewcommand{\lstlistingname}{Código}
%\renewcommand{\lstlistlistingname}{Índice de códigos}
%\usepackage{pdfpages}
%\usepackage{booktabs}
%\usepackage{x86}
\usepackage{color}
%\definecolor{dkgreen}{rgb}{0,0.6,0}
%\definecolor{gray}{rgb}{0.5,0.5,0.5}
%\definecolor{mauve}{rgb}{0.58,0,0.82}

%\lstset{ %
%	language=[x86]Assembler,       % the language of the code
%	basicstyle=\footnotesize,       % the size of the fonts that are used for the code
%	numbers=left,                   % where to put the line-numbers
%	numberstyle=\tiny\color{gray},  % the style that is used for the line-numbers
%	stepnumber=1,                   % the step between two line-numbers. If it's 1, each line 
%	 will be numbered
%	numbersep=5pt,                  % how far the line-numbers are from the code
%	backgroundcolor=\color{white},  % choose the background color. You must add \usepackage{color}
%	showspaces=false,               % show spaces adding particular underscores
%	showstringspaces=false,         % underline spaces within strings
%	showtabs=false,                 % show tabs within strings adding particular underscores
%	frame=single,                   % adds a frame around the code
%	rulecolor=\color{black},        % if not set, the frame-color may be changed on line-breaks within not-black text (e.g. commens (green here))
%	tabsize=2,                      % sets default tabsize to 2 spaces
%	captionpos=b,                   % sets the caption-position to bottom
%	breaklines=true,                % sets automatic line breaking
%	breakatwhitespace=false,        % sets if automatic breaks should only happen at whitespace
%	title=\lstname,                 % show the filename of files included with \lstinputlisting;
%	 also try caption instead of title
%	keywordstyle=\color{blue},          % keyword style
%	commentstyle=\color{dkgreen},       % comment style
%	stringstyle=\color{mauve},         % string literal style
%	escapeinside={\%*}{*)},            % if you want to add a comment within your code
%	morekeywords={*,...},               % if you want to add more keywords to the set
%	literate=
%	{á}{{\'a}}1 {é}{{\'e}}1 {í}{{\'i}}1 {ó}{{\'o}}1 {ú}{{\'u}}1
%	{Á}{{\'A}}1 {É}{{\'E}}1 {Í}{{\'I}}1 {Ó}{{\'O}}1 {Ú}{{\'U}}1
%	{à}{{\`a}}1 {è}{{\`e}}1 {ì}{{\`i}}1 {ò}{{\`o}}1 {ù}{{\`u}}1
%	{À}{{\`A}}1 {È}{{\'E}}1 {Ì}{{\`I}}1 {Ò}{{\`O}}1 {Ù}{{\`U}}1
%	{ä}{{\"a}}1 {ë}{{\"e}}1 {ï}{{\"i}}1 {ö}{{\"o}}1 {ü}{{\"u}}1
%	{Ä}{{\"A}}1 {Ë}{{\"E}}1 {Ï}{{\"I}}1 {Ö}{{\"O}}1 {Ü}{{\"U}}1
%	{â}{{\^a}}1 {ê}{{\^e}}1 {î}{{\^i}}1 {ô}{{\^o}}1 {û}{{\^u}}1
%	{Â}{{\^A}}1 {Ê}{{\^E}}1 {Î}{{\^I}}1 {Ô}{{\^O}}1 {Û}{{\^U}}1
%	{œ}{{\oe}}1 {Œ}{{\OE}}1 {æ}{{\ae}}1 {Æ}{{\AE}}1 {ß}{{\ss}}1
%	{ű}{{\H{u}}}1 {Ű}{{\H{U}}}1 {ő}{{\H{o}}}1 {Ő}{{\H{O}}}1
%	{ç}{{\c c}}1 {Ç}{{\c C}}1 {ø}{{\o}}1 {å}{{\r a}}1 {Å}{{\r A}}1
%	{€}{{\euro}}1 {£}{{\pounds}}1
%	{ñ}{{\~n}}1,
%}

\usepackage{amsmath,amsfonts,amsthm} % Math packages
\usepackage{graphics,graphicx, floatrow} %para incluir imágenes y notas en las imágenes
\usepackage{wrapfig}
\usepackage{graphics,graphicx, float} %para incluir imágenes y colocarlas

% Para hacer tablas comlejas
%\usepackage{multirow}
%\usepackage{threeparttable}

%\usepackage{sectsty} % Allows customizing section commands
%\allsectionsfont{\centering \normalfont\scshape} % Make all sections centered, the default font and small caps

\usepackage{fancyhdr} % Custom headers and footers
\pagestyle{fancyplain} % Makes all pages in the document conform to the custom headers and footers
\fancyhead{} % No page header - if you want one, create it in the same way as the footers below
\fancyfoot[L]{} % Empty left footer
\fancyfoot[C]{} % Empty center footer
\fancyfoot[R]{\thepage} % Page numbering for right footer
\renewcommand{\headrulewidth}{0pt} % Remove header underlines
\renewcommand{\footrulewidth}{0pt} % Remove footer underlines
\setlength{\headheight}{13.6pt} % Customize the height of the header

\numberwithin{equation}{section} % Number equations within sections (i.e. 1.1, 1.2, 2.1, 2.2 instead of 1, 2, 3, 4)
\numberwithin{figure}{section} % Number figures within sections (i.e. 1.1, 1.2, 2.1, 2.2 instead of 1, 2, 3, 4)
\numberwithin{table}{section} % Number tables within sections (i.e. 1.1, 1.2, 2.1, 2.2 instead of 1, 2, 3, 4)

\setlength\parindent{0pt} % Removes all indentation from paragraphs - comment this line for an assignment with lots of text

\newcommand{\horrule}[1]{\rule{\linewidth}{#1}} % Create horizontal rule command with 1 argument of height


\usepackage{titlesec}
\titleclass{\subsubsubsection}{straight}[\subsection]

\newcounter{subsubsubsection}[subsubsection]
\renewcommand\thesubsubsubsection{\thesubsubsection.\arabic{subsubsubsection}.}
%\renewcommand\theparagraph{\thesubsubsubsection.\arabic{paragraph}} % optional; useful if paragraphs are to be numbered

\titleformat{\subsubsubsection}
{\normalfont\normalsize\bfseries}{\thesubsubsubsection}{1em}{}
\titlespacing*{\subsubsubsection}
{0pt}{3.25ex plus 1ex minus .2ex}{1.5ex plus .2ex}

\makeatletter
%\renewcommand\paragraph{\@startsection{paragraph}{5}{\z@}%
%	{3.25ex \@plus1ex \@minus.2ex}%
%	{-1em}%
%	{\normalfont\normalsize\bfseries}}
%\renewcommand\subparagraph{\@startsection{subparagraph}{6}{\parindent}%
%	{3.25ex \@plus1ex \@minus .2ex}%
%	{-1em}%
%	{\normalfont\normalsize\bfseries}}
\def\toclevel@subsubsubsection{4}
%\def\toclevel@paragraph{5}
%\def\toclevel@paragraph{6}
\def\l@subsubsubsection{\@dottedtocline{4}{7em}{4em}}
%\def\l@paragraph{\@dottedtocline{5}{10em}{5em}}
%\def\l@subparagraph{\@dottedtocline{6}{14em}{6em}}
\makeatother

\setcounter{secnumdepth}{4}
\setcounter{tocdepth}{4}

\renewcommand{\thefootnote}{\arabic{footnote}}

%\usepackage{pgfgantt}

\usepackage{tocloft} % para margen en en listado de tablas y figuras
\setlength{\cftfignumwidth}{3em}
\setlength{\cfttabnumwidth}{3em}